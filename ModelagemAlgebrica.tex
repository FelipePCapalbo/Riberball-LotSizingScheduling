\section{Modelo matemático}

\section*{Conjuntos}
\begin{tabularx}{\linewidth}{@{}L{2.8cm}R@{}}
$T$ & Conjunto de períodos de planejamento (horas).\\
$I$ & Conjunto de produtos.\\
$J$ & Conjunto de máquinas.\\
\end{tabularx}

\section*{Parâmetros}
\begin{tabularx}{\linewidth}{@{}L{2.8cm}R@{}}
$D_{it}$ & Demanda do item $i$ no período $t$.\\
$c^l_{i}$ & Custo de venda perdida do item $i$.\\
$c^s_{ij}$ & Custo de \textit{setup} do item $i$ na máquina $j$.\\
$p_{ij}$ & Produtividade do item $i$ na máquina $j$.\\
$cap_{jt}$ & Capacidade disponível da máquina $j$ no período $t$.\\
$t^s_{j}$ & Tempo de \textit{setup} (horas) na máquina $j$.\\
$ss_{it}$ & Estoque de segurança do item $i$ no período $t$.\\
$I_{i0}$ & Estoque inicial do item $i$.\\
$M$ & Valor suficientemente grande (\textit{Big-M}).\\
\end{tabularx}

\section*{Variáveis}
\begin{tabularx}{\linewidth}{@{}L{2.8cm}R@{}}
$x_{ijt} \ge 0$ & Quantidade produzida do item $i$ na máquina $j$ no período $t$.\\
$y_{ijt} \in \{0,1\}$ & 1 se o item $i$ é produzido na máquina $j$ no período $t$; 0 caso contrário.\\
$s_{ijt} \in \{0,1\}$ & 1 se a máquina $j$ está configurada para o item $i$ no período $t$; 0 caso contrário.\\
$\delta_{ijt} \in \{0,1\}$ & 1 se ocorre \textit{setup} para o item $i$ na máquina $j$ no período $t$; 0 caso contrário.\\
$z_{jt} \in \{0,1\}$ & 1 se a máquina $j$ está ociosa no período $t$; 0 caso contrário.\\
$I_{it} \ge 0$ & Nível de estoque do item $i$ ao final do período $t$.\\
$K_{it} \ge 0$ & Quantidade de venda perdida do item $i$ no período $t$.\\
\end{tabularx}

\section*{\textbf{Minimizar}}

\begin{equation}
Z =
\sum_{i \in I}\sum_{j \in J}\sum_{t \in T} c^s_{ij} \delta_{ijt}
+
\sum_{i \in I}\sum_{t \in T} c^l_{i} K_{it}
\label{eq:objetivo}
\end{equation}


\noindent\textbf{Sujeito a}
\begin{align}
\sum_{i \in I} s_{ijt} &= 1 && \forall j, t \label{eq:estado_unico} \\
\delta_{ijt} &\ge s_{ijt} - s_{ij(t-1)} && \forall i, j, t \label{eq:setup_delta1} \\
\delta_{ijt} &\ge y_{ijt} - s_{ij(t-1)} && \forall i, j, t \label{eq:setup_delta2} \\
\sum_{i \in I} y_{ijt} &\le |I| (1 - z_{jt}) && \forall j, t \label{eq:ociosidade} \\
s_{ijt} &\le y_{ijt} + z_{jt} && \forall i, j, t \label{eq:link_prod_estado} \\
x_{ijt} &\le M \cdot y_{ijt} && \forall i, j, t \label{eq:ativacao} \\
\sum_{i \in I} \left( \frac{x_{ijt}}{p_{ij}} + t^s_{j} \delta_{ijt} \right) &\le cap_{jt} && \forall j, t \label{eq:capacidade} \\
I_{i(t-1)} + \sum_{j \in J} x_{ijt} &= I_{it} + D_{it} - K_{it} && \forall i, t \label{eq:balanco} \\
I_{it} &\ge ss_{it} && \forall i, t \label{eq:seguranca}
\end{align}

A função objetivo \eqref{eq:objetivo} minimiza o custo total composto por dois termos: custos de \textit{setup} das máquinas e penalidades por vendas perdidas.

A restrição \eqref{eq:estado_unico} garante que cada máquina $j$ mantenha exatamente uma configuração (estado) ativa para algum produto $i$ em cada período $t$, assegurando a continuidade do setup ("carry-over").

As restrições \eqref{eq:setup_delta1} e \eqref{eq:setup_delta2} detectam a ocorrência de \textit{setup}. A variável $\delta_{ijt}$ é forçada a 1 se a máquina muda seu estado para o produto $i$ ou se inicia a produção do produto $i$ vindo de um estado diferente no período anterior.

As restrições \eqref{eq:ociosidade} e \eqref{eq:link_prod_estado} gerenciam a ociosidade. Se a máquina é declarada ociosa ($z_{jt}=1$), nenhuma produção pode ocorrer ($\sum y_{ijt} = 0$). Caso contrário, se a máquina não está ociosa, ela deve obrigatoriamente produzir o item para o qual está configurada ($s_{ijt} \le y_{ijt}$), evitando setups "fantasmas" sem produção.

A restrição \eqref{eq:ativacao} vincula a quantidade produzida $x_{ijt}$ à variável binária de produção $y_{ijt}$.

A restrição \eqref{eq:capacidade} limita a capacidade disponível em cada máquina e período, consumida tanto pelo tempo de processamento da produção quanto pelo tempo improdutivo de \textit{setup}.

A restrição \eqref{eq:balanco} realiza o balanço de massa do estoque, considerando o estoque do período anterior e a produção atual, subtraindo a demanda atendida (Demanda - Venda Perdida).

Finalmente, a restrição \eqref{eq:seguranca} assegura que o nível de estoque final respeite o estoque de segurança mínimo estabelecido.